%% Based on a TeXnicCenter-Template by Gyorgy SZEIDL.
%%%%%%%%%%%%%%%%%%%%%%%%%%%%%%%%%%%%%%%%%%%%%%%%%%%%%%%%%%%%%

%------------------------------------------------------------
%
\documentclass{amsart}
%
%----------------------------------------------------------
% This is a sample document for the AMS LaTeX Article Class
% Class options
%        -- Point size:  8pt, 9pt, 10pt (default), 11pt, 12pt
%        -- Paper size:  letterpaper(default), a4paper
%        -- Orientation: portrait(default), landscape
%        -- Print size:  oneside, twoside(default)
%        -- Quality:     final(default), draft
%        -- Title page:  notitlepage, titlepage(default)
%        -- Start chapter on left:
%                        openright(default), openany
%        -- Columns:     onecolumn(default), twocolumn
%        -- Omit extra math features:
%                        nomath
%        -- AMSfonts:    noamsfonts
%        -- PSAMSFonts  (fewer AMSfonts sizes):
%                        psamsfonts
%        -- Equation numbering:
%                        leqno(default), reqno (equation numbers are on the right side)
%        -- Equation centering:
%                        centertags(default), tbtags
%        -- Displayed equations (centered is the default):
%                        fleqn (equations start at the same distance from the right side)
%        -- Electronic journal:
%                        e-only
%------------------------------------------------------------
% For instance the command
%          \documentclass[a4paper,12pt,reqno]{amsart}
% ensures that the paper size is a4, fonts are typeset at the size 12p
% and the equation numbers are on the right side
%
\usepackage{amsmath}%
\usepackage{amsfonts}%
\usepackage{amssymb}%
\usepackage{graphicx}
%------------------------------------------------------------
% Theorem like environments
%
\newtheorem{theorem}{Theorem}
\theoremstyle{plain}
\newtheorem{acknowledgement}{Acknowledgement}
\newtheorem{algorithm}{Algorithm}
\newtheorem{axiom}{Axiom}
\newtheorem{case}{Case}
\newtheorem{claim}{Claim}
\newtheorem{conclusion}{Conclusion}
\newtheorem{condition}{Condition}
\newtheorem{conjecture}{Conjecture}
\newtheorem{corollary}{Corollary}
\newtheorem{criterion}{Criterion}
\newtheorem{definition}{Definition}
\newtheorem{example}{Example}
\newtheorem{exercise}{Exercise}
\newtheorem{lemma}{Lemma}
\newtheorem{notation}{Notation}
\newtheorem{problem}{Problem}
\newtheorem{proposition}{Proposition}
\newtheorem{remark}{Remark}
\newtheorem{solution}{Solution}
\newtheorem{summary}{Summary}
\numberwithin{equation}{section}
%--------------------------------------------------------
\begin{document}
Introduction: Biology and transport problems\\
\\
\\
1. Ecological networks\\
	- Describe from previous work (suggestions for papers?)\\
	- food web (Is a food web optimal?)(Sun -> Grass -> Herbivore -> Carnivore -> Bacteria(?) )\\
		-Interested in statistical properties\\
		-Decide how the 'p' level is done----keep constant\\
		-Need to add more nodes (possibly adjust p,c) in this food web case to handle the complexity of interactions between different types of individuals\\
		-Concern/Difficulty: Connecting to empirical data\\
		-Check for connection weights\\
2. Network Optimization\\
	-Empirical study of networks\\
		-information transport (c >0 (?)) (see email from 12/21)\\
		-Relation to evolution\\
3. Transport problems\\
\\
\\
\\
Description of the model\\
\\
1. Discrete network model\\
 a. Pump\\
 b. Other nodes\\
 c. Connectivity rules\\
 d. Consumption of energy\\
 e. Cost of transport proportional to amount of transport\\
2. Determining energy at each node\\
 a. Given an energy that the pump receives, we can know the energy received by all other cells.\\
 b.Simple examples (figure 1)\\
 c. Critical energy - Knowing a point when a certain number of cells die, find the energy at the pump \\
3. Special Networks\\
 a. Line\\
 b. Star\\
 c. Importance of these motifs\\
 d. Simple generalizations\\
  i. Fork\\
	ii. Stellar flare (re-name)\\
 e. Provide explicit formula of first critical energy\\
 f. Node to die is independent of c and p\\
  i. only one path back to the pump\\
\\
Recursive Formula \\
\\
0. Motivation\\
 a. Where to start?\\
 b. Which node dies first? second, etc.\\
 c. More paths back to the pump as the network grows\\
 d. How do we handle loops (i.e. when there is no explicit formula)?\\
1. The formula itself\\
 a. The special topologies don't span the space of allowable topologies\\
  i. Topologies may depend on c and p (figure)\\
	ii. What happens when a node dies in general?\\
	iii. How many nodes die at a given event?\\
	iv. The formula\\
	v. Example of a simple topology with non-trivial death events (figure)\\
	vi. Example of dependence on p even within a single topology\\
 b. Calculate critical energy for everyone, the maximum is the 1st critical energy (2nd largest $->$2nd critical energy, 3rd, etc.)\\
2. (Aside/Example: Finding the stoichiometric coefficient of a chemical reaction)\\
3. Anything else goes into Supplementary Material\\
\\
Example of competing topologies\\
\\
0. Find best $E_k$. \\
 a. How to minimize a specific $E_k$ that we are more interested in\\
 b. Which of the possible topologies is the best for a given beta (environmental energy distribution)?\\
1. Analytically find when p values cross\\
2. Not reasonable to do this for large numbers of topologies\\
3. \\
\\
Colormap and Data Analysis (random and builder)\\
\\
1. Cross-section\\
 a. p\\
 b. E\\
2. Mean and Standard Deviation\\
3. Low/Upper edge\\
4. Derivatives\\
5. Compare random and builder (Description of each in supplementary material)\\
\\
Conclusions and Future Work\\
\\
0. Connections to biological evolution\\
1. Analysis of E1, E2, and E3 in real vs. optimal networks\\
2. Comparing result with Diffusion (flux?)\\
\\
\end{document}